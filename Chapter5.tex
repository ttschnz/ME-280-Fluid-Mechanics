\chapter{Control Volume Approach using RTT}

Recall the three principal physical laws:
\begin{itemize}
	\setlength{\itemsep}{-5pt}
	\item Mass conservation
	\item Newton's \nth2 law
	\item energy conservation
\end{itemize}
These are all expressed on systems, but we would like to apply them for control volumes. We can use RTT to do this.

\section{Mass Conservation}
The arbitrary quantity $B$ is the mass $M$. This implies that $b=1$. We plug $B=M$ into eq. \eqref{eq:rtt_final} and find:
\begin{equation}
	\begin{split}
		\frac{dM_{sys}}{Dt} = \boxed{\frac{\partial}{\partial t}  \int_{cv} \rho dV + \int _{cs} \rho \vec v \cdot \hat n \,dA =0}\\
	\end{split}
	\label{eq:continuity}
\end{equation}
Which is the continuity equation. In words, this means:
\begin{equation*}
	\left(\substack{
		\text{The rate of change}\\
		\text{of the mass in the}\\
		\text{coincident system}
		}\right) = \left(\substack{
		\text{The rate of change}\\
		\text{in the coincident}\\
		\text{control volume}
		}\right) + \left(\substack{
		\text{The net rate of flow}\\
		\text{of mass through}\\
		\text{the control surface}
		}\right)
\end{equation*}

\section{Momentum: Newtons \nth{2} Law}

\section{Energy Conservation}