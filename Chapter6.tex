\chapter{Differential Analysis}

\section{Mass Conservation}
{\footnotesize\textit{Munson 6.2}}
Recall the law for mass conservation that was transferred to a control volume approach in the last chapter:
\begin{equation*}
	\begin{split}
		\frac{dM_{sys}}{Dt}&=0\\
		\text{\footnotesize Reynolds Transport Theorem:} \quad \frac{\partial}{\partial t}\int_{cv} \rho \,dV + \underbrace{\int_{cs} \rho \vec v \cdot \hat n\,dA}_{\stackrel{\text{Gauss}}{=} \int_{cv} \nabla \cdot (\rho \vec v )\,dV} &= 0\hfill\\
		\int_{cv} \left[\frac{\partial}{\partial} \rho + \nabla \cdot (\rho \vec v)\right]\,dV &= 0 \stackrel{(*)}\Longleftrightarrow \frac{\partial}{\partial} \rho + \nabla \cdot (\rho \vec v)=0
	\end{split}
\end{equation*}
Where at $(*)$ we figure that an integral over an arbitrary volume can only be zero if the integrand itself is zero.
The resulting equation is what we call the \textbf{continuity equation}:
\begin{equation*}
	\boxed{\frac{\partial \rho}{\partial t} + \nabla \cdot \left(\rho \vec v \right) = 0}\qquad \text{Eularian Form}
\end{equation*}
The Lagrangian form is as follows:
\begin{equation*}
	\boxed{\frac{D\rho}{Dt} + \rho \nabla \cdot \vec v = 0} \qquad \text{Lagrangian Form}
\end{equation*}

For an incompressible case, where $\rho = const$, then 
\begin{equation*}
	\boxed{\nabla \cdot \vec v = 0}  \qquad \text{Incompressible}
\end{equation*}




\section{Navier-Stokes Equations (Momentum Conservation / Newtons \nth2 law)}
{\footnotesize\textit{Munson 6.3}}  Recall newtons \nth2 law for a Lagrangian system, which we rewrote to an eularian fashion:
\begin{equation*}
	\frac{D}{Dt} (M\vec v)_{sys} = \sum \vec F_{ext}\implies \frac{\partial}{\partial t} \int_{cv}\rho\vec v \,dV + \int_{cs}\rho \vec v (\vec v \cdot \hat n) = \sum \vec F_{ext}
\end{equation*}
As before, we would like to rewrite everything under one volume integral. 

The second term can be rewritten using Gauss:
\begin{equation*}
	\int_{cs} \rho \vec v(\vec v \cdot \hat n) \,dA \stackrel{\text{Gauss}}{=} \int_{cv} \nabla\cdot (\vec v \rho \vec v)\,dV
\end{equation*}
Since $\rho$ is constant, we can denote the $i$th component of the integrand as 
\begin{equation*}
	\nabla\cdot (\vec v \rho \vec v)\stackrel{\nabla \cdot \vec v = 0}= \rho v_j \partial_j v_i = \rho \vec v \cdot \nabla \vec v
\end{equation*}

For a control volume, we want to consider the following forces (right hand term):
\begin{itemize}
	\setlength{\itemsep}{-5pt}
	\item \textbf{body forces}
	\begin{equation*}
		\int_{cv} \rho \vec g \,dV
	\end{equation*}
	\item \textbf{surface forces} (including non-viscous forces)
	
	The non-viscous force is pressure acting on the differential element. non-viscous forces are represented in the rank two stress tensor $\overline{\overline{\tau}}$:
	\begin{equation*}
		\overline{\overline{\tau}} = \begin{pmatrix}
			\tau_{xx}&\tau_{xy}&\tau_{xz}\\
			\tau_{yx}&\tau_{yy}&\tau_{yz}\\
			\tau_{zx}&\tau_{zy}&\tau_{zz}\\
		\end{pmatrix}
	\end{equation*}
	The stress tensor has the following properties:
	\begin{itemize}
		\item $\hat n \cdot \overline{\overline{\tau}}$ or $(\hat n \cdot \overline{\overline{\tau}})_j = n_i\tau_{ij}$ is the stress on the surface with normal $\hat n$.
		\item $\tau_{ij} = \tau_{ji}\implies \hat n \cdot \overline{\overline{\tau}}  = \overline{\overline{\tau}} \hat n$ (symmetric)
	\end{itemize}
	We define the cauchy stress tensor $\overline{\overline \sigma}\cdot \hat n$ with $\overline{\overline \sigma} = -p\overline{\overline{I_d}} + \overline{\overline{\tau}}$ All surface forces combined can therefore be expressed as
	\begin{equation*}
		\int_{cs} \overline{\overline \sigma }\cdot \hat n  \,dA = \int_{cv} \nabla \cdot \overline{\overline {\sigma}}\,dV
	\end{equation*}
\end{itemize}

The complete volume integral with all terms is
\begin{equation*}
	\int_{cv}\rho\frac{\partial \vec v}{\partial t} + \rho \vec v \cdot \nabla \vec v - \rho \vec g - \nabla \cdot (-p\overline{\overline{I_d}}+ \overline{\overline{\tau}}) \,dV = 0
\end{equation*}
Which is valid for all control volumes, and therefore the integrand must also be zero:
\resizebox{\textwidth}{!}{\shadowbox{$\displaystyle
	{\boxed{\rho \left(\frac{\partial \vec v}{\partial t} + \vec v \cdot \nabla \vec v \right) =-\nabla p + \rho \vec g + \nabla \cdot \overline{\overline{\tau}}}} \qquad \substack{\text{Cauchy's Equation }\\\text{(Incompressible flow)}}
	$}}